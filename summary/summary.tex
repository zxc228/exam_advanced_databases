\documentclass[a4paper,12pt]{article}

% Русский и английский язык
\usepackage[utf8]{inputenc}
\usepackage[T2A]{fontenc}
\usepackage[russian,english]{babel}
\usepackage{enumitem}
% Код
\usepackage{listings}
\usepackage{xcolor}
\usepackage{tabularx}

% Настройка стиля кода
\lstset{
    language=Python, % Язык по умолчанию
    basicstyle=\ttfamily\footnotesize,
    keywordstyle=\color{blue},
    commentstyle=\color{gray},
    stringstyle=\color{red},
    breaklines=true,
    frame=single,
    numbers=left,
    numberstyle=\tiny\color{gray},
    captionpos=b,
}

% Заголовок документа
\title{Summary about MongoDB, Redis, Neo4j, MapReduce}
\author{Author: Ilya Istomin}
\date{\today}

\begin{document}

\maketitle

\section{Введение}
This document contains notes on MongoDB, Redis, Neo4j databases, MapReduce and Casandra concept.



\section{MongoDB}
MongoDB — нереляционная база данных, ориентированная на документы.

\subsection{CRUD commands in MongoDB}
\subsubsection{Insert document (Create)}
Вставляет один документ в коллекцию users.
\newline
Inserts a single document into the users collection.

\begin{lstlisting}[language=SQL]
    db.users.insertOne({
        "_id": 11,
        "firstName": "Lisa",
        "lastName": "Wong",
        "email": "lisa.wong@example.com",
        "age": 30,
        "username": "lisaw",
        "lastLogin": "2024-10-01"
      })

\end{lstlisting}

\subsubsection{Insert many documents (Create)}
Вставляет несколько документов в коллекцию products.
\newline
Inserts multiple documents into the products collection.
\begin{lstlisting}[language=SQL]
db.products.insertMany([
  { "_id": 11, "name": "Tablet", "description": "Portable tablet with HD screen", "price": 600, "category": "Electronics" },
  { "_id": 12, "name": "Microwave", "description": "Compact microwave oven", "price": 90, "category": "Home Appliances" }
])
\end{lstlisting}

\subsubsection{Find document (Read)}
Ищет и возвращает один документ из users по \_id.
\newline
Finds and returns a single document from users by \_id.

\begin{lstlisting}[language=SQL]
db.users.findOne({ "_id": 1 })
\end{lstlisting}

\subsubsection{Find many documents (Read)}
Ищет всех пользователей, чей возраст больше 30 лет.
\newline
Finds all users whose age is greater than 30.

\begin{lstlisting}[language=SQL]
db.users.find({ "age": { $gt: 30 } })
\end{lstlisting}


\subsubsection{Update document (Update)}
Обновляет документ в коллекции users.
\newline    
Updates a document in the users collection.
\begin{lstlisting}[language=SQL]
db.users.updateOne(
    { "_id": 1 },
    { $set: { "email": "john.doe.new@example.com" } }
    )
      
\end{lstlisting}

\subsubsection{Update many documents (Update)}
Увеличивает цену всех продуктов на 10\%.
\newline
Increases the price of all products by 10\%.
\begin{lstlisting}[language=SQL]
db.products.updateMany(
    {},
    { $mul: { "price": 1.1 } }
)
\end{lstlisting}

\subsubsection{Upsert document (Update)}
Обновляет документ в коллекции users или вставляет новый документ, если документ не найден.
\newline
Updates a document in the users collection or inserts a new document if the document is not found.
\begin{lstlisting}[language=SQL]
db.users.updateOne(
  { "username": "newuser" },  
  { 
    $set: { 
      "firstName": "New",
      "lastName": "User",
      "email": "new.user@example.com",
      "age": 25,
      "lastLogin": "2024-11-01"
    } 
  },
  { upsert: true }
)

\end{lstlisting}

\subsubsection{Delete one document (Delete)}
Удаляет один документ из коллекции products.
\newline
Deletes a single document from the products collection.

\begin{lstlisting}[language=SQL]
db.products.deleteOne({ "_id": 1 })
\end{lstlisting}

\subsubsection{Delete many documents (Delete)}
Удаляет всех пользователей, которые не входили в систему после 1 мая 2024 года.
\newline
Deletes all users who have not logged in after May 1, 2024.

\begin{lstlisting}[language=SQL]
db.users.deleteMany(
  { "lastLogin": { $lt: "2024-05-01" } }
)
\end{lstlisting}

\subsubsection{Indexes}
Индексы это структуры данных, которые ускоряют поиск документов в коллекции.

Составной индекс это индекс, который содержит несколько полей.

\begin{lstlisting}[language=SQL]
db.users.createIndex({ "firstName": 1, "lastName": 1 })
\end{lstlisting}

Текстовый индекс это индекс, который позволяет выполнять текстовый поиск в полях строкового типа.

\begin{lstlisting}[language=SQL]
db.products.createIndex({ "description": "text" })
\end{lstlisting}

Уникальный индекс это индекс, который гарантирует уникальность значений в поле или полях.
\begin{lstlisting}[language=SQL]
db.users.createIndex({ "username": 1 }, { unique: true })
\end{lstlisting}


\subsubsection{Text search}
Полнотекстовый поиск позволяет выполнять поиск по текстовым полям.
\newline
Full-text search allows you to search text fields.
\begin{lstlisting}[language=SQL]
db.products.find({ $text: { $search: "oven" } })
\end{lstlisting}


Поиск с условием AND.
\newline
Search with AND condition.
\begin{lstlisting}[language=SQL]
db.products.find({
    $and: [
        { "price": { $gt: 50 } },
        { $text: { $search: "oven" } }
    ]
    })
\end{lstlisting}

\subsubsection{Geo queries}
Создание гео индекса.
\newline
Create a geo index.
\begin{lstlisting}[language=SQL]
db.places.createIndex({ "location": "2dsphere" })
\end{lstlisting}

Поиск всех мест, находящихся в радиусе 5 км от координат
\newline
Find all places within a 5 km radius of the coordinates.
\begin{lstlisting}[language=SQL]
db.places.find({
"location": {
    $near: {
    $geometry: { type: "Point", coordinates: [-73.9857, 40.7484] },
    $maxDistance: 5000  
    }
}
})
\end{lstlisting}
Поиск всех мест, отсортированных по близости к координатам.
\newline 
Find all places sorted by proximity to the coordinates.
\begin{lstlisting}[language=SQL]
db.places.find({
    "location": {
    $near: {
        $geometry: { type: "Point", coordinates: [-73.9857, 40.7484] }
    }
    }
})
\end{lstlisting}

\subsubsection{Logical operators}
Поиск мест с использованием оператора OR.
\newline
Find places using the OR operator.

\begin{lstlisting}[language=SQL]
db.places.find({
    $or: [
        { "name": { $regex: "the", $options: "i" } },
        { "name": { $regex: "of", $options: "i" } }
    ]
})
\end{lstlisting}

Поиск продуктов в категориях "Electronics" или "Furniture".
\newline
Find products in the "Electronics" or "Furniture" categories.

\begin{lstlisting}[language=SQL]
db.products.find({
    "category": { $in: ["Electronics", "Furniture"] }
})
\end{lstlisting}

Поиск мест у которых обе координаты меньше 0.
\newline
Find places where both coordinates are less than 0.
\begin{lstlisting}[language=SQL]
db.places.find({
    "location.coordinates.0": { $lt: 0 },
    "location.coordinates.1": { $lt: 0 }
})
      
\end{lstlisting}

\subsubsection{Aggregation}
Все ключи аггрегации


\subsubsection{Фильтрация и выборка данных}
\begin{itemize}[noitemsep]
    \item \textbf{\$match}: Фильтрация документов (аналог find())
    \item \textbf{\$project}: Выбор полей и создание вычисляемых значений
    \item \textbf{\$limit}: Ограничение количества документов
    \item \textbf{\$skip}: Пропуск определенного количества документов
\end{itemize}

\subsubsection{Группировка и сортировка}
\begin{itemize}[noitemsep]
    \item \textbf{\$group}: Группировка документов (аналог GROUP BY)
    \item \textbf{\$sort}: Сортировка результатов
\end{itemize}

\subsubsection{Работа с массивами}
\begin{itemize}[noitemsep]
    \item \textbf{\$unwind}: Разворачивает массив (каждый элемент создаёт отдельный документ)
\end{itemize}

\subsubsection{Операции с несколькими коллекциями}
\begin{itemize}[noitemsep]
    \item \textbf{\$lookup}: Объединение данных из разных коллекций (аналог JOIN)
    \item \textbf{\$graphLookup}: Рекурсивный поиск (например, в древовидных структурах)
\end{itemize}

\subsubsection{Вывод результатов}
\begin{itemize}[noitemsep]
    \item \textbf{\$out}: Записывает результат в новую коллекцию
    \item \textbf{\$merge}: Обновляет или добавляет данные в существующую коллекцию
\end{itemize}

\subsubsection{Операторы агрегации}

\subsubsection{Математические операторы}
\begin{itemize}[noitemsep]
    \item \textbf{\$sum}: Суммирует значения
    \item \textbf{\$avg}: Вычисляет среднее значение
    \item \textbf{\$min}: Находит минимальное значение
    \item \textbf{\$max}: Находит максимальное значение
    \item \textbf{\$add}: Сложение чисел
    \item \textbf{\$subtract}: Вычитание
    \item \textbf{\$multiply}: Умножение
    \item \textbf{\$divide}: Деление
    \item \textbf{\$mod}: Остаток от деления
\end{itemize}

\subsubsection{Операторы строк}
\begin{itemize}[noitemsep]
    \item \textbf{\$concat}: Объединяет строки
    \item \textbf{\$substr}: Извлекает подстроку
    \item \textbf{\$toUpper}: Преобразует в верхний регистр
    \item \textbf{\$toLower}: Преобразует в нижний регистр
\end{itemize}

\subsubsection{Операторы массивов}
\begin{itemize}[noitemsep]
    \item \textbf{\$push}: Добавляет элементы в массив
    \item \textbf{\$addToSet}: Добавляет уникальные элементы в массив
    \item \textbf{\$size}: Возвращает размер массива
    \item \textbf{\$filter}: Фильтрует массив
\end{itemize}

\subsubsection{Операторы условий}
\begin{itemize}[noitemsep]
    \item \textbf{\$cond}: Если-иначе (аналог if-else)
    \item \textbf{\$ifNull}: Возвращает значение, если не null
\end{itemize}

\subsubsection{Логические операторы}
\begin{itemize}[noitemsep]
    \item \textbf{\$and}: Логическое И
    \item \textbf{\$or}: Логическое ИЛИ
    \item \textbf{\$not}: Логическое НЕ
\end{itemize}

\subsubsection{Операторы даты}
\begin{itemize}[noitemsep]
    \item \textbf{\$year}: Получает год из даты
    \item \textbf{\$month}: Получает месяц
    \item \textbf{\$dayOfMonth}: Получает день месяца
    \item \textbf{\$hour}: Получает час
    \item \textbf{\$second}: Получает секунду
\end{itemize}



\newpage
\section{Redis}
Redis — это хранилище данных типа "ключ-значение".

\subsection{Пример команды в Redis}
\begin{lstlisting}[language=sh, caption=Добавление значения]
SET user:1 "John Doe"
\end{lstlisting}

\section{Neo4j}
Neo4j — графовая база данных.

\subsection{Пример запроса в Neo4j}
\begin{lstlisting}[language=SQL, caption=Создание узла]
CREATE (p:Person {name: "Alice"})
\end{lstlisting}

\section{MapReduce}
MapReduce — модель обработки больших данных.

\subsection{Пример псевдокода}
\begin{lstlisting}[language=Python, caption=MapReduce]
def map(key, value):
    for word in value.split():
        emit(word, 1)

def reduce(key, values):
    emit(key, sum(values))
\end{lstlisting}


\section{Cassandra}
Cassandra — это распределённая NoSQL база данных, предназначенная для обработки больших объёмов данных, обеспечивающая высокую доступность и целостность данных.

\subsection{Основные характеристики}
\begin{itemize}
    \item Децентрализация (высокая доступность)
    \item Репликация
    \item Отказоустойчивость
    \item Масштабируемость
\end{itemize}

\subsection{Архитектура Cassandra}
\subsubsection{Основные элементы}
\begin{itemize}
    \item \textbf{Узел (Node)} – отдельный экземпляр Cassandra.
    \item \textbf{Раздел (Partition)} – базовая единица информации, важная для репликации и организации данных.
    \item \textbf{Стойка (Rack)} – логическая группа узлов.
    \item \textbf{Дата-центр (Data Center)} – логическая группа стоек.
    \item \textbf{Кластер (Cluster)} – полный набор узлов (структура полного кольца токенов).
\end{itemize}

\subsubsection{Разделение данных (Partitioning)}
\begin{itemize}
    \item Partition – базовая единица хранения.
    \item Partition Key – ключ, определяющий расположение данных в базе.
    \item Partitioner – механизм распределения данных по узлам на основе ключа.
    \item Кольцо токенов (Token Ring) – распределение разделов между узлами.
\end{itemize}

\subsubsection{Виртуальные узлы (vNodes)}
\begin{itemize}
    \item В Cassandra 2.0 узлы делятся на виртуальные подузлы.
    \item Основные процессы:
    \begin{itemize}
        \item \textbf{Bootstrap} – автоматическое перераспределение сегментов токенов при добавлении нового узла.
        \item \textbf{Decommissioning} – перераспределение сегментов при удалении узла.
    \end{itemize}
\end{itemize}

\subsubsection{Репликация (Replication)}
Cassandra обеспечивает высокую доступность и отказоустойчивость за счёт репликации данных.
\begin{itemize}
    \item \textbf{Replication Factor} – количество узлов, хранящих копии данных.
    \item \textbf{Replication Strategy} – стратегия распределения реплик:
    \begin{itemize}
        \item SimpleStrategy – последовательное распределение реплик.
        \item NetworkTopologyStrategy – реплики распределяются по разным дата-центрам.
    \end{itemize}
\end{itemize}

\subsubsection{Уровень согласованности (Consistency Level)}
Определяет, сколько узлов должны ответить на запрос до того, как клиент получит подтверждение.
\begin{itemize}
    \item \textbf{Варианты уровней согласованности}:
    \begin{itemize}
        \item ONE, TWO, THREE – ответ от 1, 2 или 3 узлов.
        \item ANY – ответ от любого узла.
        \item ALL – ответ от всех узлов.
        \item QUORUM – 51% узлов.
    \end{itemize}
\end{itemize}

\subsection{Поток записи (Write Path)}
Cassandra выполняет запись данных в несколько этапов:
\begin{enumerate}
    \item \textbf{Memtable (в памяти)} – временное хранение записей.
    \item \textbf{CommitLog (на диск)} – защита от потери данных.
    \item \textbf{SSTable (на диск)} – окончательная запись.
    \item \textbf{Компактизация} – объединение SSTables для оптимизации запросов.
\end{enumerate}

\subsection{Поток чтения (Read Path)}
Cassandra выполняет чтение данных по следующему процессу:
\begin{enumerate}
    \item Поиск в Memtable.
    \item Поиск в SSTable (используется Partition Index и Partition Summary).
    \item Использование Bloom Filter для ускоренного поиска.
    \item Слияние данных (merge) и возврат клиенту.
\end{enumerate}

\subsection{Репликация данных и согласованность}
Cassandra использует репликацию, чтобы данные были доступны даже при сбоях.
\begin{itemize}
    \item \textbf{Hinted Handoff} – временное хранение изменений до восстановления узла.
    \item \textbf{Read Repair} – обновление устаревших данных при чтении.
    \item \textbf{Anti-Entropy Repair} – процесс фоновой синхронизации данных между узлами.
\end{itemize}


\end{document}
